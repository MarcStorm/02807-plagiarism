\section{Introduction}

% What is plagiarism?
% Why do you want to do detect this?
% Who are the users?
% Only searching for whole articles that are similar or example is much easier that to check if a given document as copied small sections from many different documents.

\emph{Wikipedia} is often used as a source of information. Unfortunately it is quite common that the information from the articles is used in reports, research papers etc. without citation or without being modified sufficiently and incorporated in the existing text. This is known as plagiarism which is defined as \emph{\say{an act or instance of using or closely imitating the language and thoughts of another author without authorisation and the representation of that author's work as one's own, as by not crediting the original author}}\footnote{\url{https://www.dictionary.com/browse/plagiarism}}

% Introduce and define what problem was chosen and why
This paper presents a solution for finding candidate where the text contents of a query document could be plagiarised from. The candidates, should there be any for a query document, will be found from the English Wikipedia articles. The possible plagiarised articles are found by utilising the Locality-Sensitive Hashing algorithm and data structure, FINISH THE INTRODUCTION